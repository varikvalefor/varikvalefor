\documentclass{article}
\usepackage{amsthm}
\usepackage{parskip}
\newtheorem{thm}{Theorem}
\newtheorem{axiom}{Axiom}
\title{le mupti ja krinu be lo nu cimoi prenu pe'a ciska}
\author{la .varik.\ .valefor.\ noi ponse la'o termri.\ $<$varikvalefor@aol.com$>$ .termri}
\begin{document}
\maketitle

\section{le torveki}
ni'o lo nu la .varik.\ mu'oi gy.\ third person .gy. ciska cu zmadu lo nu la .varik.\ mu'oi gy.\ first person .gy. ja mu'oi gy.\ second person .gy. ciska kei le ka se nelci la .varik.\ kei ki'u
\begin{itemize}
	\item le nu la .varik.\ to'e nelci lo prenu basyvla noi se vasru lo ke mu'oi gy.\ first person .gy. ja mu'oi gy.\ second person .gy. ke'e selci'a kei .e
	\item le nu la .varik.\ jinvi le du'u lo mu'oi gy.\ third person .gy. selci'a cu zmadu lo ke mu'oi gy.\ first person .gy. ke'e selci'a ku le ka cnici
	\item le nu lo nu mu'oi gy.\ third person .gy. ciska cu zmadu lo nu mu'oi gy.\ first person .gy. ciska kei le ka frili la .varik.
\end{itemize}

\section{lo claxu be lo prenu basyvla}
ni'o lo nu la .varik.\ cimoi prenu pe'a ciska cu zmadu lo nu la .varik.\ ke pamoi ja remoi ke'e prenu pe'a ciska kei le ka se nelci la .varik.\ kei ki'u le nu lo nu pilno lo prenu basyvla cu na sarcu lo nu cimoi prenu pe'a ciska

.i ro da poi jufra zo'u ganai da ke pamoi ja remoi ke'e prenu pe'a jufra gi da vasru lo prenu basyvla  .i ku'i naku ro da poi jufra zo'u ganai da cimoi prenu pe'a jufra gi da vasru lo prenu basyvla

\subsection{le krinu be le nu sarcu fa lo nu lo pamoi jabo remoi jufra cu vasru lo prenu basyvla}
\begin{thm}
	ni'o ro da poi se ke pamoi prenu pe'a jufra ke'e zo'u da vasru lo prenu basyvla
\end{thm}
\begin{proof}
	${}$

	.i ro da poi se ke pamoi prenu pe'a jufra ke'e zo'u da vasru lo pamoi prenu pe'a basyvla

	.i ro da poi pamoi prenu pe'a basyvla zo'u da cu prenu basyvla

	.ija'e ro da poi se ke pamoi prenu pe'a jufra ke'e zo'u da vasru lo prenu basyvla
\end{proof}
\begin{thm}
	ni'o ro da poi se ke remoi prenu pe'a jufra ke'e zo'u da vasru lo prenu basyvla
\end{thm}
\begin{proof}
	${ }$

	.i ro da poi se ke remoi prenu pe'a jufra ke'e zo'u da vasru lo remoi prenu pe'a basyvla

	.i ro da poi remoi prenu pe'a basyvla zo'u da cu prenu basyvla

	.ija'e ro da poi se ke remoi prenu pe'a jufra ke'e zo'u da vasru lo prenu basyvla
\end{proof}
\begin{axiom}
	ni'o ro da poi se ke cimoi prenu pe'a jufra ke'e zo'u na sarcu fa lo nu da vasru lo prenu basyvla
\end{axiom}

\section{le ka prenu nelrai}
ni'o li ni'e ni lo nu cimoi prenu pe'a ciska cu frili la .varik.\ cu zmadu li ni'e ni lo nu pamoi jabo remoi prenu pe'a ciska cu frili la .varik.

\subsection{le cumki krinu}

\subsubsection{lo nu zbasu lo datnynoi}
ni'o le nu la .varik.\ djica lo nu cumki fa lo nu lo jibykansa be la .varik.\ ku .e la .varik.\ cu kansi'u lo nu zbasu lo datnynoi cu mukti lo nu la .varik.\ zbasu lo cimoi prenu pe'a datnynoi  .i cumki fa le nu le nu ro da zo'u lo nu rapli da cu filri'a lo nu da binxo lo tcaci cu rinka le nu lo nu cimoi prenu pe'a ciska cu binxo lo tcaci be la .varik.

\subsubsection{le ka nelrai le cmaci bangu}
\begin{thm}
	ni'o cumki fa le nu le nu la .varik.\ se tcaci lo nu la'o gy.\ third person .gy. ciska kei kei se krinu le nu la .varik.\ cafne baupli le cmaci bangu
\end{thm}
\begin{proof}
	${ }$

	.i la .varik.\ cu prenu

	.i cafne fa lo nu la .varik.\ ciska lo jufra fi le cmaci bangu

	.i ro da poi jufra fi le cmaci bangu zo'u da cu cimoi prenu pe'a

	.i ro da poi prenu zo'u ro de zo'u ganai cafne fa lo nu da co'e de gi cumfi fa le nu lo nu co'e de cu binxo le tcaci be da

	.ija'e cumki fa le nu le nu la .varik.\ se tcaci lo nu la'o gy.\ third person .gy. ciska kei kei se krinu le nu la .varik.\ cafne baupli le cmaci bangu
\end{proof}
\end{document}
