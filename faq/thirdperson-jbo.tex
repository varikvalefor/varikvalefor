\documentclass{article}
\usepackage{amsthm}
\usepackage{parskip}
\usepackage[margin=1.25in]{geometry}
\newtheorem{thm}{Theorem}
\newtheorem{axiom}{Axiom}
\title{le mupti ja krinu be lo nu cimoi prenu pe'a ciska}
\author{Varik Valefor\\{\small $<$varikvalefor@aol.com$>$}}
\begin{document}
\maketitle
\section{le torveki}
li ni'e ni la .varik.\ nelci lo nu la .varik.\ me la'o gy.\ third person .gy ciska cu zmadu li ni'e ni la .varik.\ nelci lo nu la .varik.\ me la'o gy.\ first person .gy ja me la'o gy.\ second person .gy ciska kei kei ki'u
\begin{itemize}
	\item le nu la .varik.\ to'e nelci lo prenu basyvla noi se vasru lo se ke me la'o gy.\ first person .gy ja me la'o gy.\ second person .gy ciska ke'e kei .e
	\item le nu la .varik.\ jinvi le du'u li ni'e ni lo se ke me la'o gy.\ third person .gy ciska ke'e cu cnici cu zmadu li ni'e ni lo se ke me la'o gy.\ second person .gy jabo me la'o gy.\ first person .gy ciska ke'e cu cnici kei .e
	\item le nu li ni'e ni lo nu me la'o gy.\ third person .gy ciska cu frili la .varik.\ cu zmadu li ni'e ni lo nu me la'o gy.\ first person .gy jebo me la'o gy.\ second person .gy ciska cu frili la .varik.
\end{itemize}
\section{lo claxu be lo prenu basyvla}
.ni'o li ni'e ni la .varik.\ nelci lo nu la .varik.\ cimoi prenu pe'a ciska cu zmadu li ni'e ni la .varik.\ nelci lo nu la .varik.\ pamoi jabo remoi prenu pe'a ciska pagbu se krinu le nu ganai la .varik.\ cimoi prenu pe'a ciska gi na sarcu fa lo nu la .varik.\ pilno lo prenu basyvla

.i ro da poi se ke pamoi jabo remoi prenu pe'a jufra ke'e zo'u sarcu fa le nu da vasru lo prenu basyvla  .i ku'i ro da poi se ke cimoi prenu pe'a jufra ke'e zo'u na sarcu fa le nu da vasru lo prenu basyvla
\subsection{le krinu be le nu sarcu fa lo nu lo pamoi jabo remoi jufra cu vasru lo prenu basyvla}
\begin{thm}
	.ni'o ro da poi se ke pamoi prenu pe'a jufra ke'e zo'u da vasru lo prenu basyvla
\end{thm}
\begin{proof}
	${}$

	.i ro da poi se ke pamoi prenu pe'a jufra ke'e zo'u da vasru lo pamoi prenu pe'a basyvla

	.i ro da poi pamoi prenu pe'a basyvla zo'u da cu prenu basyvla

	.ija'e ro da poi se ke pamoi prenu pe'a jufra ke'e zo'u da vasru lo prenu basyvla
\end{proof}
\begin{thm}
	.ni'o ro da poi se ke remoi prenu pe'a jufra ke'e zo'u da vasru lo prenu basyvla
\end{thm}
\begin{proof}
	${ }$

	.i ro da poi se ke remoi prenu pe'a jufra ke'e zo'u da vasru lo remoi prenu pe'a basyvla

	.i ro da poi remoi prenu pe'a basyvla zo'u da cu prenu basyvla

	.ija'e ro da poi se ke remoi prenu pe'a jufra ke'e zo'u da vasru lo prenu basyvla
\end{proof}
\begin{axiom}
	.ni'o ro da poi se ke cimoi prenu pe'a jufra ke'e zo'u na sarcu fa lo nu da vasru lo prenu basyvla
\end{axiom}
\section{le ka prenu nelrai}
.ni'o li ni'e ni lo nu cimoi prenu pe'a ciska cu frili la .varik.\ cu zmadu li ni'e ni lo nu pamoi jabo remoi prenu pe'a ciska cu frili la .varik.
\subsection{le cumki krinu}
\subsubsection{lo nu zbasu lo datnynoi}
.ni'o le nu la .varik.\ djica lo nu cumki fa lo nu lo jibykansa be la .varik.\ ku .e la .varik.\ cu kansi'u lo nu zbasu lo datnynoi cu mukti lo nu la .varik.\ zbasu lo cimoi prenu pe'a datnynoi  .i cumki fa le nu le nu ro da zo'u lo nu rapli da cu filri'a lo nu da binxo lo tcaci cu rinka le nu lo nu cimoi prenu pe'a ciska cu binxo lo tcaci be la .varik.
\subsubsection{le ka nelrai le cmaci bangu}
\begin{thm}
	.ni'o cumki fa le nu le nu la .varik.\ se tcaci lo nu la'o gy.\ third person .gy ciska kei kei se krinu le nu la .varik.\ cafne baupli le cmaci bangu
\end{thm}
\begin{proof}
	${ }$

	.i la .varik.\ cu prenu

	.i cafne fa lo nu la .varik.\ ciska lo jufra fi le cmaci bangu

	.i ro da poi jufra fi le cmaci bangu zo'u da cu cimoi prenu pe'a

	.i ro da poi prenu zo'u ro de zo'u ganai cafne fa lo nu da co'e de gi cumfi fa le nu lo nu co'e de cu binxo le tcaci be da

	.ija'e cumki fa le nu le nu la .varik.\ se tcaci lo nu la'o gy.\ third person .gy ciska kei kei se krinu le nu la .varik.\ cafne baupli le cmaci bangu
\end{proof}
\end{document}
